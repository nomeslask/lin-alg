\ifx\inkludert\undefined
\documentclass[norsk,a4paper,twocolumn,oneside]{memoir}

\usepackage[utf8]{inputenc}
\usepackage{babel}
\usepackage{amsmath,amssymb,amsthm}
\usepackage{mathrsfs}
\usepackage[total={17cm,27cm}]{geometry}
\usepackage[table]{xcolor}
%\usepackage{tabularx}
\usepackage{systeme}
\usepackage{hyperref}
%\usepackage{enumerate}
\usepackage{ifthen}
\usepackage{textgreek}
\usepackage{multirow}
\usepackage{placeins}
\usepackage{caption}
\usepackage{lmodern}

%\usepackage{sectsty}
\setsecheadstyle{\bfseries\large}
%\subsectionfont{\bf\normalsize}

\usepackage{tikz,pgfplots}
\usetikzlibrary{calc}
\usetikzlibrary{arrows.meta}
\def\centerarc[#1](#2)(#3:#4:#5)% Syntax: [draw options] (center) (initial angle:final angle:radius)
    { \draw[#1] ($(#2)+({#5*cos(#3)},{#5*sin(#3)})$) arc (#3:#4:#5); }
\usepackage{pgfornament}

\newcommand{\defterm}[1]{\emph{#1}}

\newcommand{\N}{\mathbb{N}}
\newcommand{\Z}{\mathbb{Z}}
\newcommand{\Q}{\mathbb{Q}}
\newcommand{\R}{\mathbb{R}}
\newcommand{\C}{\mathbb{C}}

\newcommand{\M}{\mathcal{M}} % vektorrom av matriser
\newcommand{\Cf}{\mathcal{C}} % vektorrom av kontinuerlige funksjoner
\renewcommand{\P}{\mathcal{P}} % vektorrom av polynomer
\newcommand{\B}{\mathscr{B}} % basis

\renewcommand{\Im}{\operatorname{Im}}
\renewcommand{\Re}{\operatorname{Re}}

\newcommand{\abs}[1]{|#1|}
\newcommand{\intersect}{\cap}
\newcommand{\union}{\cup}
\newcommand{\fcomp}{\circ}
\newcommand{\iso}{\cong}

\newcommand{\roweq}{\sim}
\DeclareMathOperator{\Sp}{Sp}
\DeclareMathOperator{\Null}{Null}
\DeclareMathOperator{\Col}{Col}
\DeclareMathOperator{\Row}{Row}
\DeclareMathOperator{\rank}{rank}
\DeclareMathOperator{\im}{im}
\DeclareMathOperator{\id}{id}
\DeclareMathOperator{\Hom}{Hom}
\newcommand{\tr}{^\top}
\newcommand{\koord}[2]{[\,{#1}\,]_{#2}} % koordinater mhp basis

\newcommand{\V}[1]{\mathbf{#1}}
\newcommand{\vv}[2]{\begin{bmatrix} #1 \\ #2 \end{bmatrix}}
\newcommand{\vvS}[2]{\left[ \begin{smallmatrix} #1 \\ #2 \end{smallmatrix} \right]}
\newcommand{\vvv}[3]{\begin{bmatrix} #1 \\ #2 \\ #3 \end{bmatrix}}
\newcommand{\vvvv}[4]{\begin{bmatrix} #1 \\ #2 \\ #3 \\ #4 \end{bmatrix}}
\newcommand{\vvvvv}[5]{\begin{bmatrix} #1 \\ #2 \\ #3 \\ #4 \\ #5 \end{bmatrix}}
\newcommand{\vn}[2]{\vvvv{#1_1}{#1_2}{\vdots}{#1_#2}}

\newcommand{\e}{\V{e}}
\renewcommand{\u}{\V{u}}
\renewcommand{\v}{\V{v}}
\newcommand{\w}{\V{w}}
\renewcommand{\a}{\V{a}}
\renewcommand{\b}{\V{b}}
\newcommand{\x}{\V{x}}
\newcommand{\0}{\V{0}}

\newenvironment{amatrix}[1]{% "augmented matrix"
  \left[\begin{array}{*{#1}{c}|c}
}{%
  \end{array}\right]
}

\newcommand{\boks}[1]{\framebox{\strut $#1$}}

% \newcounter{notatnr}
% \newcommand{\notatnr}[2]
% {\setcounter{notatnr}{#1}%
%  \setcounter{page}{#2}%
% }

\newtheorem{thm}{Teorem}[chapter]
\newtheorem{fishythm}[thm]{«Teorem»}
\newtheorem*{thm-nn}{Teorem}
\newtheorem{cor}[thm]{Korollar}
\newtheorem{lem}[thm]{Lemma}
\newtheorem{prop}[thm]{Proposisjon}
\theoremstyle{definition}
\newtheorem{exx}[thm]{Eksempel}
\newtheorem*{defnx}{Definisjon}
\newtheorem*{oppg}{Oppgave}
\newtheorem*{merkx}{Merk}
\newtheorem*{kommentarx}{Kommentar}
\newtheorem*{spmx}{Spørsmål}

\newenvironment{defn}
  {\pushQED{\qed}\renewcommand{\qedsymbol}{$\triangle$}\defnx}
  {\popQED\enddefnx}
\newenvironment{ex}
  {\pushQED{\qed}\renewcommand{\qedsymbol}{$\triangle$}\exx}
  {\popQED\endexx}
\newenvironment{merk}
  {\pushQED{\qed}\renewcommand{\qedsymbol}{$\triangle$}\merkx}
  {\popQED\endmerkx}
\newenvironment{spm}
  {\pushQED{\qed}\renewcommand{\qedsymbol}{$\triangle$}\spmx}
  {\popQED\endspmx}

\setlength{\columnsep}{26pt}

\newcommand{\linje}{%
\begin{center}
\rule{.8\linewidth}{0.4pt}
\end{center}
}


\newcommand{\chapternumber}{}

\makechapterstyle{tma4110}{%
 \renewcommand*{\chapterheadstart}{}
 \renewcommand*{\printchaptername}{}
 \renewcommand*{\chapternamenum}{}
 \renewcommand*{\printchapternum}{\renewcommand{\chapternumber}{\thechapter}}
 \renewcommand*{\afterchapternum}{}
 \renewcommand*{\printchapternonum}{\renewcommand{\chapternumber}{}}
 \renewcommand*{\printchaptertitle}[1]{
\LARGE
\begin{tabularx}{\textwidth}{cXr}
\cellcolor{black}\color{white}%
\textbf{\chapternumber} &
\textbf{##1}
\hfill &
\footnotesize% kan legge til tekst her hvis vi vil
\\ \hline
\end{tabularx}%
}
 \renewcommand*{\afterchaptertitle}{\par\nobreak\vskip \afterchapskip}
 % \newcommand{\chapnamefont}{\normalfont\huge\bfseries}
 % \newcommand{\chapnumfont}{\normalfont\huge\bfseries}
 % \newcommand{\chaptitlefont}{\normalfont\Huge\bfseries}
 \setlength{\beforechapskip}{0pt}
 \setlength{\midchapskip}{0pt}
 \setlength{\afterchapskip}{10pt}
}
\chapterstyle{tma4110}
\pagestyle{plain}


\newboolean{vis-oppgaver}
\newboolean{vis-losninger}
\setboolean{vis-oppgaver}{true}
\setboolean{vis-losninger}{false}

\newcounter{oppg-kap} % kapittelnummerering for oppgaver
\newcounter{oppgnr}[oppg-kap]
\newcounter{punktnr}[oppgnr]

\newenvironment{oppgave}%
 {\ifthenelse{\boolean{vis-oppgaver}}%
             {\par\noindent\stepcounter{oppgnr}\textbf{\arabic{oppgnr}.}}%
             {\expandafter\comment}}%
 {\ifthenelse{\boolean{vis-oppgaver}}%
             {\par\bigskip}%
             {\expandafter\endcomment}}

\newenvironment{losning}%
 {\ifthenelse{\boolean{vis-losninger}}%
             {\par\noindent\stepcounter{oppgnr}\textbf{\arabic{oppg-kap}.\arabic{oppgnr}.}}%
             {\expandafter\comment}}%
 {\ifthenelse{\boolean{vis-losninger}}%
             {\par\bigskip}%
             {\expandafter\endcomment}}

\newenvironment{punkt}
 {\par\smallskip\noindent\stepcounter{punktnr}\textbf{\alph{punktnr})} }
 {\par}

\newcommand{\kap}[1]{\setcounter{oppg-kap}{#1}\addtocounter{oppg-kap}{-1}\stepcounter{oppg-kap}}

\newcommand{\oppgaver}[1]{%
  \kap{#1}%
  \ifthenelse{\boolean{vis-oppgaver}}%
             {\linje\section*{Oppgaver}}%
             {}}

\usepackage{xr}
\externaldocument{tma4110-2018h}
\newcommand{\kapittel}[2]{\setcounter{chapter}{#1}\addtocounter{chapter}{-1}\chapter{#2}}
\newcommand{\kapittelslutt}{\enddocument}
\begin{document}
\chapterstyle{tma4110}
\pagestyle{plain}
\fi


\chapter*{Introduksjon}
\addcontentsline{toc}{chapter}{Introduksjon}
\label{ch:introduksjon}

Velkommen til emnet TMA4110 Matematikk 3.

Vi som underviser emnet høsten 2018 har valgt å skrive notater som
inneholder det vi gjennomgår i forelesningene, og resultatet er det du
leser nå.  Pensum for emnet er det som står i disse notatene --
hverken mer eller mindre.

Vi begynner med å ta et overblikk over temaene vi skal innom i løpet
av semesteret.  Det gjør ikke noe om du ikke forstår alt i denne
introduksjonen nå; vi skal gjennomgå det i detalj siden.  Men
forhåpentligvis får du en idé om hva emnet inneholder.

\smallskip

Emnet er delt inn i tre separate deler som egentlig hører til helt
forskjellige områder innenfor matematikk:
\begin{enumerate}
\item Lineær algebra
\item Komplekse tall
\item Lineære differensiallikninger
\end{enumerate}
Delen om lineær algebra utgjør hoveddelen av emnet og er den vi vil
bruke mest tid på.

\smallskip

Grunnen til at disse tilsynelatende urelaterte temaene er gruppert
sammen i ett emne er at det er noen viktige tilknytningspunkter mellom
dem som gjør at det er fornuftig å lære om dem sammen.


\section*{Lineær algebra}

Algebra er et område innenfor matematikk som i utgangspunktet handler
om å løse likninger.  For å få en idé om hva algebra er for noe, kan
det være nyttig å vite hva det \emph{ikke} er.  Her er en liten guide
til hvordan noen av de tingene du antagelig har gjort i ditt
matematiske liv så langt passer inn i ulike områder innen matematikk:

Hvis du deriverer eller integrerer, så driver du med \emph{analyse}.
Hvis du tegner en figur, driver du antagelig med \emph{geometri}.
Hvis du teller antall måter du kan plukke opp forskjellige fargede
kuler fra en pose på, så driver du med \emph{kombinatorikk}; men hvis
du deretter regner ut sannsynligheten for at du får to røde kuler, så
har du flyttet deg over til \emph{sannsynlighetsregning}.  Hvis du
prøver å forstå reglene for hvordan et matematisk bevis utføres, så
driver du med \emph{logikk}.  Og hvis du løser en likning, da driver
du med algebra.

\smallskip

Da du for eksempel (en gang for mange år siden) lærte at en
andregradslikning
\[
ax^2 + bx + c = 0
\]
kan løses ved hjelp av formelen
\[
x = \frac{-b \pm \sqrt{b^2 - 4ac}}{2a},
\]
så var det en del av den klassiske, grunnleggende algebraen du lærte.

\smallskip

Ordet «algebra» kommer av det arabiske \emph{al-jabr}, som var en del
av tittelen på en bok skrevet omkring år 820 av den persiske
matematikeren al-Khwarizmi.  Denne boken inneholdt metoder for å løse
likninger (blant annet andregradslikningen nevnt over), og la
grunnlaget for algebra som fagområde.

Men så er det slik med matematikk at den stadig endrer seg.
Matematisk forskning er en evig runddans av at matematikere finner opp
nye teknikker og konsepter for å finne svar på spørsmål de synes er
interessante, og så finner de ut at man kan stille nye interessante
spørsmål om de nye tingene de har funnet opp, og så har man det
gående.

I løpet av 1800-tallet og begynnelsen på 1900-tallet førte den
matematiske utviklingen til at algebraen som fagfelt skiftet fokus.
Man utviklet visse former for matematiske strukturer som ble
brukt til å forstå ulike typer likninger, og over tid fant man ut at
disse strukturene kunne generaliseres og være nyttige også til andre
ting enn løsing av likninger.  Dermed endte man opp med forskjellige
typer \emph{abstrakte algebraiske strukturer} (noen av disse kalles
\emph{grupper}, \emph{ringer} og \emph{moduler}), og algebra i dag
handler primært om å forstå disse strukturene.  Denne nye formen for
algebra kalles \emph{abstrakt algebra} (eller \emph{moderne algebra}),
for å skille den fra den mer tradisjonelle algebraen som handler om
løsing av likninger.

\medskip
Greit, det var veldig mye prat om algebra.  Men hva er \emph{lineær
  algebra} for noe?

Algebra handler opprinnelig om likninger, og lineær algebra er den
delen av algebraen som handler om lineære likninger.

En lineær likning med én ukjent ser generelt slik ut:
\[
ax = b
\]
Dersom $a \ne 0$, kan vi løse denne likningen ved å dele på $a$:
\[
x = b/a
\]
Og det er egentlig omtrent alt som er å si om lineære likninger med én
ukjent.

Men hvis vi ser på et system av flere lineære likninger, med flere
ukjente, blir det straks mer interessant.  Her er et eksempel på et
system av lineære likninger:
\[
\systeme{
  3x + 5y - 2z = 14,
   x - 9y + 7z = 0,
 -5x + 4y + 8z = -3
}
\]
Studiet av slike systemer er utgangspunktet for lineær algebra.

\smallskip
Det første vi skal lære er en metode (som kalles
\emph{gausseliminasjon}) for å løse lineære likningssystemer på en
effektiv måte.

Deretter skal vi se at ved å innføre noen nye konsepter -- nemlig
\emph{vektorer} og \emph{matriser} -- kan vi få en bedre forståelse av
lineære likningssystemer.

Du har antagelig hørt om vektorer før, og antagelig har du lært å
tenke på en vektor som en pil -- noe som har en lengde og en retning.
\begin{center}
\begin{tikzpicture}[scale=.9]
\draw[->] (0,1) -- (3,0);
\end{tikzpicture}
\\
{\small \textit{En pil}}
\end{center}
Dette er imidlertid bare ett aspekt av vektorer.  Vi vil velge å se på
et \emph{punkt} i planet, og \emph{pilen} fra origo til dette punktet, og
\emph{koordinatene} til punktet, som tre forskjellige representasjoner
av den samme vektoren.
\begin{center}
\begin{tikzpicture}[scale=.9]
\draw[->] (-3.5,0) -- (3.5,0);
\draw[->] (0,-1.2) -- (0,2.4);
\foreach \x in {-3,-2,-1,1,2,3}
\draw (\x cm,1pt) -- (\x cm,-1pt) node[anchor=north] {$\x$};
\foreach \y in {-1,1,2}
\draw (1pt,\y cm) -- (-1pt,\y cm) node[anchor=east] {$\y$};
\draw[-{>[length=7,width=5]},line width=1pt] (0,0) -- (2,1);
\filldraw (2,1) circle [radius=2pt];
\node[anchor=west] at (2,1) {$(2,1)$};
\end{tikzpicture}
\\
{\small \textit{Vektorenes treenighet:\\punkt -- pil -- koordinater}}
\end{center}
Koordinatene som angir en vektor, for eksempel $(2,1)$, vil vi
vanligvis skrive på denne måten:
\[
\vv{2}{1}
\]
Dette kaller vi en \emph{kolonnevektor} (vi kan også si
\emph{søylevektor}).

Det er klart at vi kan se på vektorer i to dimensjoner eller i tre
dimensjoner -- altså i et plan eller i et rom.  Men hvis vi tenker på
vektorer bare som kolonnevektorer, og glemmer det med punkter og
piler, så er det ikke noe i veien for å snakke om vektorer i fire
dimensjoner, eller fem, eller så mange dimensjoner vi vil.  Og det
skal vi gjøre.

Men hva har dette med lineære likningssystemer å gjøre?  Vi skal
definere aritmetiske operasjoner for vektorer (på ganske naturlige
måter), slik at likningssystemet vi så for en stund siden kan skrives
på denne måten:
\[
\vvv{3}{1}{-5} \cdot x + \vvv{5}{-9}{4} \cdot y + \vvv{-2}{7}{8} \cdot z = \vvv{14}{0}{-3}
\]
Istedenfor et system med flere likninger har vi altså én likning, der
koeffisientene og høyresiden er vektorer.  Istedenfor å tenke på
problemet som det å finne tall som er løsninger av flere forskjellige
likninger, tenker vi på det som å finne ut om visse vektorer kan
kombineres slik at vi får en viss annen vektor.

Så innfører vi matriser, som er rektangulære tabeller med tall, og vi
kan skrive om systemet vårt til:
\[
\begin{bmatrix}
 3 &  5 & -2 \\
 1 & -9 &  7 \\
-5 &  4 &  8
\end{bmatrix}
\vvv{x}{y}{z}
= \vvv{14}{0}{-3}
\]
Her har vi en matrise ganger en vektor på venstresiden, og en vektor
på høyresiden.  Når vi har lært om matriser, kan vi skrive et generelt
lineært likningssystem på den konsise formen
\[
A \V{x} = \V{b},
\]
der $A$ er en matrise, $\V{b}$ er en konstant vektor, og $\V{x}$ er en
ukjent vektor.

Men det stopper ikke der!  Vi kan snu likheten $A \V{x} = \V{b}$ litt
på hodet.  Istedenfor å si at $\V{x}$ er en ukjent som vi vil finne,
så kan vi si at når matrisen $A$ ganges med en vektor~$\V{x}$, så får
vi ut en ny vektor~$\V{b}$.  Med andre ord: En matrise gir opphav til
en funksjon som tar inn vektorer og gir ut vektorer.  Hvis matrisen
har $m$~rader og $n$~kolonner, så kan vi bruke den til å lage en
funksjon~$T$ som tar inn $n$-dimensjonale vektorer og gir ut
$m$-dimensjonale vektorer.  En slik funksjon kalles en
\emph{lineærtransformasjon}, og vi skriver:
\[
T \colon \R^n \to \R^m
\]
Her står $\R^n$ for mengden av alle $n$-dimensjonale kolonnevektorer,
og en slik mengde kalles et \emph{vektorrom}.

Nå viser det seg at vektorrom, lineærtransformasjoner og matriser er
interessante ting å studere i seg selv, og at vi med utgangspunkt i
disse tingene kan bygge opp en stor og flott matematisk teori med
anvendelsesområder som går langt ut over det å løse likningssystemer.
Den teorien er lineær algebra.


\section*{Komplekse tall}

Hva er et tall?  Da du som barn lærte å telle, lærte du tallene
\[
1,\ 2,\ 3,\ 4,\ \ldots
\]
Så lærte du å legge sammen tall, og å trekke tall fra hverandre.  Da
viste det seg at det går an å stille spørsmål som ikke har svar: Hva
er $3 - 5$?

Men så lærte du at slike spørsmål likevel kan besvares når man bare
har innført noen nye tall:
\[
0,\ -1,\ -2,\ -3,\ \ldots
\]

Videre utvidet du tallforståelsen din med \emph{rasjonale tall}
(brøker av heltall, for eksempel $7/5$), og så lærte du at det også
finnes tall som ikke er rasjonale, for eksempel $\sqrt{2}$ og $\pi$.
Når vi tar med alle slike tall, har vi mengden av \emph{reelle tall}.

Fremdeles er det spørsmål som ikke kan besvares, for eksempel: Hva er
$\sqrt{-1}$?  Det finnes ikke noe reelt tall som vi kan opphøye i
andre og få $-1$.

Men denne situasjonen er egentlig helt tilsvarende som da vi bare
kjente til de positive tallene og lurte på hva $3 - 5$ er.  Akkurat
som vi da kunne utvide tallsystemet ved å legge til negative tall, kan
vi nå legge til nye tall som gjør at uttrykket $\sqrt{-1}$ gir mening.

Vi lager et nytt tall, som vi kaller~$i$, og som er definert til å
være slik at
\[
i^2 = -1
\]
For å få et tallsystem som oppfører seg slik det skal, må vi kunne
gange og legge sammen $i$ med et hvilket som helst annet tall.  Det
gjør at vi må ha med flere nye tall, og til sammen får vi at alle tall
som kan skrives som
\[
a + bi
\qquad
\text{(der $a$ og~$b$ er reelle tall)}
\]
må være med i det nye tallsystemet.  Disse tallene kalles
\emph{komplekse tall}.

Mengden av reelle tall visualiserer vi som en uendelig lang tallinje.
Mengden av komplekse tall visualiserer vi som et todimensjonalt plan.
\begin{center}
\begin{tikzpicture}[scale=.9]
\draw[->] (-3.3,0) -- (3.5,0);
\draw[->] (0,-1.3) -- (0,2.4);
\foreach \x in {-3,-2,-1,1,2,3}
\draw (\x cm,1pt) -- (\x cm,-1pt) node[anchor=north] {$\x$};
\draw (1pt, 2 cm) -- (-1pt, 2 cm) node[anchor=east] {$2i$};
\draw (1pt, 1 cm) -- (-1pt, 1 cm) node[anchor=east] {$i$};
\draw (1pt,-1 cm) -- (-1pt,-1 cm) node[anchor=east] {$-i$};
\filldraw (1,2) circle [radius=2pt];
\node[anchor=west] at (1,2) {$(1 + 2i)$};
\end{tikzpicture}
\\
{\small \textit{Det komplekse planet}}
\end{center}

Du må ikke la deg lure av navnet til å tro at komplekse tall er
kompliserte å ha med å gjøre.  På mange måter er det enklere å jobbe
med komplekse tall enn med reelle tall.

Innenfor de komplekse tallene kan vi for eksempel alltid ta
kvadratrøtter av hvilke som helst tall, uten å måtte tenke på om de er
negative.  Med andre ord: alle likninger på formen $x^2 = a$ har
løsninger.  Og ikke bare det, men alle andregradslikninger
\[
a x^2 + bx + c = 0
\]
har løsninger.  Og ikke bare det, men alle polynomlikninger av hvilken
som helst grad, altså alle likninger på formen
\[
a_n x^n + a_{n-1} x^{n-1} + \cdots + a_1 x + a_0 = 0,
\]
har løsninger.

Det som imidlertid kan gjøre komplekse tall litt vanskelige er at de
ikke helt passer inn i vår intuitive forståelse av hva et tall skal
være.  Med reelle tall kan vi lett se for oss hvordan et tall kan
representere noe målbart i den virkelige verden: en avstand, et areal,
en hastighet eller liknende.  Med komplekse tall er det vanskeligere å
se for seg hva tallene kan representere.

Når vi får bruk for komplekse tall er det ofte slik at vi starter med
noe som bare handler om reelle tall, og får et sluttsvar med bare
reelle tall, men må innom de komplekse tallene i mellomregningen.  Vi
kommer til å se eksempler på nettopp dette når vi i emnets siste del
skal løse differensiallikninger.


\section*{Lineære differensiallikninger}

En \emph{differensiallikning} er en likning der den ukjente er en
funksjon, og der den deriverte av denne ukjente funksjonen også er med
i likningen.

Vi skal se på to forskjellige typer differensiallikninger.  Den ene
typen er lineære andreordens differensiallikninger, som vil si
likninger på formen
\[
y'' + p y' + q y = g,
\]
der $y$ er en ukjent funksjon av en variabel~$t$, og $g$ er en kjent
funksjon av~$t$, og $p$ og~$q$ er konstanter.

Når vi skal løse en slik likning får vi bruk for å lage en
«hjelpelikning», nemlig andregradslikningen
\[
\lambda^2 + p \lambda + q = 0
\]
der $\lambda$ er den ukjente, og $p$ og~$q$ er de samme konstantene
som vi hadde i differensiallikningen.  Det viser seg nemlig at
løsningene av denne likningen gir oss viktig informasjon om hvordan
løsningene av differensiallikningen ser ut.  Men avhengig av hva $p$
og~$q$ er, er det ikke sikkert at denne andregradslikningen har noen
løsning i reelle tall.  Her blir vi reddet av at vi har lært om
komplekse tall!  Vi kan alltid finne komplekse tall som er løsninger
av hjelpelikningen vår, og disse kan vi igjen bruke til å finne
løsningene av differensiallikningen vi startet med.

\smallskip
Den andre typen differensiallikninger vi skal se på er systemer av
førsteordens lineære differensiallikninger, som vil si
likningssystemer på formen
\[
\left\{
\begin{aligned}
x_1' &= a_{11} x_1 + a_{12} x_2 + \cdots + a_{1n} x_n \\
x_2' &= a_{21} x_1 + a_{22} x_2 + \cdots + a_{2n} x_n \\
     &\ \ \vdots \\
x_n' &= a_{n1} x_1 + a_{n2} x_2 + \cdots + a_{nn} x_n
\end{aligned}
\right.
\]
der koeffisientene $a_{ij}$ er konstanter og hver $x_i$ er en ukjent
funksjon av~$t$.

Når vi har lært om lineær algebra og matriser, ser vi at et slikt
system også kan skrives på den mer kompakte formen
\[
\V{x}' = A \V{x},
\]
der $\V{x}$ er en vektorfunksjon og $A$ er en matrise.  For å løse
systemet vil vi bruke avansert lineær-algebraisk magi som litt
forenklet sagt går ut på å vri det $n$-dimensjonale rommet om til et
nytt $n$-dimensjonalt rom der systemet blir enkelt å løse, så løse det
der, og til slutt vri rommet tilbake og ta løsningene med oss.

\kapittelslutt
