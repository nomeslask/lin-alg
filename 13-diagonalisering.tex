\ifx\inkludert\undefined
\documentclass[norsk,a4paper,twocolumn,oneside]{memoir}

\usepackage[utf8]{inputenc}
\usepackage{babel}
\usepackage{amsmath,amssymb,amsthm}
\usepackage{mathrsfs}
\usepackage[total={17cm,27cm}]{geometry}
\usepackage[table]{xcolor}
%\usepackage{tabularx}
\usepackage{systeme}
\usepackage{hyperref}
%\usepackage{enumerate}
\usepackage{ifthen}
\usepackage{textgreek}
\usepackage{multirow}
\usepackage{placeins}
\usepackage{caption}
\usepackage{lmodern}

%\usepackage{sectsty}
\setsecheadstyle{\bfseries\large}
%\subsectionfont{\bf\normalsize}

\usepackage{tikz,pgfplots}
\usetikzlibrary{calc}
\usetikzlibrary{arrows.meta}
\def\centerarc[#1](#2)(#3:#4:#5)% Syntax: [draw options] (center) (initial angle:final angle:radius)
    { \draw[#1] ($(#2)+({#5*cos(#3)},{#5*sin(#3)})$) arc (#3:#4:#5); }
\usepackage{pgfornament}

\newcommand{\defterm}[1]{\emph{#1}}

\newcommand{\N}{\mathbb{N}}
\newcommand{\Z}{\mathbb{Z}}
\newcommand{\Q}{\mathbb{Q}}
\newcommand{\R}{\mathbb{R}}
\newcommand{\C}{\mathbb{C}}

\newcommand{\M}{\mathcal{M}} % vektorrom av matriser
\newcommand{\Cf}{\mathcal{C}} % vektorrom av kontinuerlige funksjoner
\renewcommand{\P}{\mathcal{P}} % vektorrom av polynomer
\newcommand{\B}{\mathscr{B}} % basis

\renewcommand{\Im}{\operatorname{Im}}
\renewcommand{\Re}{\operatorname{Re}}

\newcommand{\abs}[1]{|#1|}
\newcommand{\intersect}{\cap}
\newcommand{\union}{\cup}
\newcommand{\fcomp}{\circ}
\newcommand{\iso}{\cong}

\newcommand{\roweq}{\sim}
\DeclareMathOperator{\Sp}{Sp}
\DeclareMathOperator{\Null}{Null}
\DeclareMathOperator{\Col}{Col}
\DeclareMathOperator{\Row}{Row}
\DeclareMathOperator{\rank}{rank}
\DeclareMathOperator{\im}{im}
\DeclareMathOperator{\id}{id}
\DeclareMathOperator{\Hom}{Hom}
\newcommand{\tr}{^\top}
\newcommand{\koord}[2]{[\,{#1}\,]_{#2}} % koordinater mhp basis

\newcommand{\V}[1]{\mathbf{#1}}
\newcommand{\vv}[2]{\begin{bmatrix} #1 \\ #2 \end{bmatrix}}
\newcommand{\vvS}[2]{\left[ \begin{smallmatrix} #1 \\ #2 \end{smallmatrix} \right]}
\newcommand{\vvv}[3]{\begin{bmatrix} #1 \\ #2 \\ #3 \end{bmatrix}}
\newcommand{\vvvv}[4]{\begin{bmatrix} #1 \\ #2 \\ #3 \\ #4 \end{bmatrix}}
\newcommand{\vvvvv}[5]{\begin{bmatrix} #1 \\ #2 \\ #3 \\ #4 \\ #5 \end{bmatrix}}
\newcommand{\vn}[2]{\vvvv{#1_1}{#1_2}{\vdots}{#1_#2}}

\newcommand{\e}{\V{e}}
\renewcommand{\u}{\V{u}}
\renewcommand{\v}{\V{v}}
\newcommand{\w}{\V{w}}
\renewcommand{\a}{\V{a}}
\renewcommand{\b}{\V{b}}
\newcommand{\x}{\V{x}}
\newcommand{\0}{\V{0}}

\newenvironment{amatrix}[1]{% "augmented matrix"
  \left[\begin{array}{*{#1}{c}|c}
}{%
  \end{array}\right]
}

\newcommand{\boks}[1]{\framebox{\strut $#1$}}

% \newcounter{notatnr}
% \newcommand{\notatnr}[2]
% {\setcounter{notatnr}{#1}%
%  \setcounter{page}{#2}%
% }

\newtheorem{thm}{Teorem}[chapter]
\newtheorem{fishythm}[thm]{«Teorem»}
\newtheorem*{thm-nn}{Teorem}
\newtheorem{cor}[thm]{Korollar}
\newtheorem{lem}[thm]{Lemma}
\newtheorem{prop}[thm]{Proposisjon}
\theoremstyle{definition}
\newtheorem{exx}[thm]{Eksempel}
\newtheorem*{defnx}{Definisjon}
\newtheorem*{oppg}{Oppgave}
\newtheorem*{merkx}{Merk}
\newtheorem*{kommentarx}{Kommentar}
\newtheorem*{spmx}{Spørsmål}

\newenvironment{defn}
  {\pushQED{\qed}\renewcommand{\qedsymbol}{$\triangle$}\defnx}
  {\popQED\enddefnx}
\newenvironment{ex}
  {\pushQED{\qed}\renewcommand{\qedsymbol}{$\triangle$}\exx}
  {\popQED\endexx}
\newenvironment{merk}
  {\pushQED{\qed}\renewcommand{\qedsymbol}{$\triangle$}\merkx}
  {\popQED\endmerkx}
\newenvironment{spm}
  {\pushQED{\qed}\renewcommand{\qedsymbol}{$\triangle$}\spmx}
  {\popQED\endspmx}

\setlength{\columnsep}{26pt}

\newcommand{\linje}{%
\begin{center}
\rule{.8\linewidth}{0.4pt}
\end{center}
}


\newcommand{\chapternumber}{}

\makechapterstyle{tma4110}{%
 \renewcommand*{\chapterheadstart}{}
 \renewcommand*{\printchaptername}{}
 \renewcommand*{\chapternamenum}{}
 \renewcommand*{\printchapternum}{\renewcommand{\chapternumber}{\thechapter}}
 \renewcommand*{\afterchapternum}{}
 \renewcommand*{\printchapternonum}{\renewcommand{\chapternumber}{}}
 \renewcommand*{\printchaptertitle}[1]{
\LARGE
\begin{tabularx}{\textwidth}{cXr}
\cellcolor{black}\color{white}%
\textbf{\chapternumber} &
\textbf{##1}
\hfill &
\footnotesize% kan legge til tekst her hvis vi vil
\\ \hline
\end{tabularx}%
}
 \renewcommand*{\afterchaptertitle}{\par\nobreak\vskip \afterchapskip}
 % \newcommand{\chapnamefont}{\normalfont\huge\bfseries}
 % \newcommand{\chapnumfont}{\normalfont\huge\bfseries}
 % \newcommand{\chaptitlefont}{\normalfont\Huge\bfseries}
 \setlength{\beforechapskip}{0pt}
 \setlength{\midchapskip}{0pt}
 \setlength{\afterchapskip}{10pt}
}
\chapterstyle{tma4110}
\pagestyle{plain}


\newboolean{vis-oppgaver}
\newboolean{vis-losninger}
\setboolean{vis-oppgaver}{true}
\setboolean{vis-losninger}{false}

\newcounter{oppg-kap} % kapittelnummerering for oppgaver
\newcounter{oppgnr}[oppg-kap]
\newcounter{punktnr}[oppgnr]

\newenvironment{oppgave}%
 {\ifthenelse{\boolean{vis-oppgaver}}%
             {\par\noindent\stepcounter{oppgnr}\textbf{\arabic{oppgnr}.}}%
             {\expandafter\comment}}%
 {\ifthenelse{\boolean{vis-oppgaver}}%
             {\par\bigskip}%
             {\expandafter\endcomment}}

\newenvironment{losning}%
 {\ifthenelse{\boolean{vis-losninger}}%
             {\par\noindent\stepcounter{oppgnr}\textbf{\arabic{oppg-kap}.\arabic{oppgnr}.}}%
             {\expandafter\comment}}%
 {\ifthenelse{\boolean{vis-losninger}}%
             {\par\bigskip}%
             {\expandafter\endcomment}}

\newenvironment{punkt}
 {\par\smallskip\noindent\stepcounter{punktnr}\textbf{\alph{punktnr})} }
 {\par}

\newcommand{\kap}[1]{\setcounter{oppg-kap}{#1}\addtocounter{oppg-kap}{-1}\stepcounter{oppg-kap}}

\newcommand{\oppgaver}[1]{%
  \kap{#1}%
  \ifthenelse{\boolean{vis-oppgaver}}%
             {\linje\section*{Oppgaver}}%
             {}}

\usepackage{xr}
\externaldocument{tma4110-2018h}
\newcommand{\kapittel}[2]{\setcounter{chapter}{#1}\addtocounter{chapter}{-1}\chapter{#2}}
\newcommand{\kapittelslutt}{\enddocument}
\begin{document}
\chapterstyle{tma4110}
\pagestyle{plain}
\fi


\kapittel{13}{Diagonalisering}
\label{ch:diagonalisering}
Dette kapitlet avslutter det skal lære om egenverdier og egenvektorer.


\section*{Hvorfor heter det diagonalisering?}
La $A$ være en  $n \times n$-matrise med $m$ lineært uavhengige egenvektorer
$\V{v}_1$, $\V{v}_2\dots \V{v}_m$ 
og tilhørende egenverdier $\lambda_1$, $\lambda_2$,$\cdots$, $\lambda_m$, . 
For hver egenvektor gjelder
\[
A\V{v}_k=\lambda_k \V{v}_k.
\]
Disse $m$ ligningene kan like gjerne organiseres i en matriseligning
\[
AV=VD, 
\]
der 
\[
D=
\begin{bmatrix}
\lambda_1      & 0      & 0      & \cdots & 0 \\
0      & \lambda_2      & 0      & \cdots & 0 \\
0      & 0      & \lambda_3      & \cdots & 0 \\
\vdots & \vdots & \vdots & \ddots & \vdots \\
0      & 0      & 0      & \cdots & \lambda_m
\end{bmatrix}
\]
og 
\[
V=
\begin{bmatrix}
\V{v}_1 & \V{v}_2 & \cdots & \V{v}_m
\end{bmatrix}.
\]

Siden $\V{v}_1$, $\V{v}_2\dots \V{v}_n$ er lineært uavhengige, er matrisen $V$  invertibel dersom $m=n$.
Vi ganger med $V^{-1}$ fra høyre og får
\[
V^{-1}AV=D.
\]
Denne operasjonen kalles å \emph{diagonalisere} $A$, og dette er grunnen til at en $n\times n$-matrise med $n$ lineært uavhengige egenvektorer kalles diagonaliserbar. 
Man kan også se det motsatt. Dersom 
\[
V^{-1}AV=D
\]
for en inverterbar $n \times n$-matrise $V$, kan vi gange fra venstre med $V$, og få
\[
AV=VD
\]
Vi ser av denne likningen at kolonnene til $V$ utgjør $n$ lineært uavhengige vektorer for $A$.

\begin{thm}
Vi kan skrive 
\[
V^{-1}AV=D
\]
hvis og bare hvis $A$ har $n$ lineært uavhengige egenvektorer.
\end{thm}

\begin{ex}	
Vi diagonaliserer matrisen 
\[
A=
\begin{bmatrix}
2     & 1 \\
1      & 2
\end{bmatrix}.
\]
Egenvektorene er 
\[
\begin{bmatrix}
1 \\ 1
\end{bmatrix}
\quad \text{og} \quad
\begin{bmatrix}
1 \\ -1
\end{bmatrix},
\]
med respektive egenverdier 3 og 1. Vi definerer
\[
V=
\begin{bmatrix}
1 & 1 \\ 1 &-1
\end{bmatrix},
\]
og beregner 
\[
V^{-1}=
\frac{1}{2}
\begin{bmatrix}
1 & 1 \\ 1 &-1
\end{bmatrix}.
\]
Vi dobbeltsjekker ved å beregne produktet
\begin{align*}
V^{-1}AV&=
\frac{1}{2}
\begin{bmatrix}
1 & 1 \\ 1 &-1
\end{bmatrix}
\begin{bmatrix}
2     & 1 \\
1      & 2
\end{bmatrix}
\begin{bmatrix}
1 & 1 \\ 1 &-1
\end{bmatrix}
\\& =
\begin{bmatrix}
3     & 0 \\
0      & 1
\end{bmatrix}
=
D \qedhere
\end{align*}
\end{ex}


\begin{ex}
Matrisen 
\[
\begin{bmatrix}
1 & 1 & 0\\  0 &1 & 0 \\ 0 & 0 & 2
\end{bmatrix}
\]
har bare to lineært uavhengige egenvektorer, og er følgelig ikke diagonaliserbar.
\end{ex}

Man kan også snu likningen
\[
D=V^{-1}AV
\]
om, og få
\[
A=VDV^{-1},
\]
og dermed tenke på diagonalisering som en faktorisering av $A$.

\begin{ex}
Vi kan faktorisere matrisen 
\[
\begin{bmatrix}
2     & 1 \\
1      & 2
\end{bmatrix}
\]
som
\[
\begin{bmatrix}
2     & 1 \\
1      & 2
\end{bmatrix}
=
\begin{bmatrix}
1 & 1 \\ 1 &-1
\end{bmatrix}
\begin{bmatrix}
3     & 0 \\
0      & 1
\end{bmatrix}
\frac{1}{2}
\begin{bmatrix}
1 & 1 \\ 1 &-1
\end{bmatrix}.\qedhere
\]
\end{ex}
\begin{merkx}
I eksemplet over spiller ikke plasseringen av $V$ og $V^{-1}$ noen rolle. Mer om dette under.
\end{merkx}

\section*{Kvadratiske former}
La $\V v$ være en kolonnevektor i $\C^n$ og $A$ en $n\times n$-matrise. En kvadratisk form er et uttrykk på formen 
\[
\V v^* A \V v
\]
Et slikt uttrykk er spesielt interessant dersom $\V v$ er en egenvektor med lengde 1 og egenverdi $\lambda$:
\[
\V v^* A \V v=\V v^* \lambda \V v=\lambda \V v^* \V v=\lambda.
\]

\begin{ex}
Vi lar 
\[
A=
\begin{bmatrix}
2     & 1 \\
1      & 2
\end{bmatrix}
\]
og 
\[
\V v=
\frac{1}{\sqrt{2}}
\begin{bmatrix}
1    \\
1    
\end{bmatrix}
\]
slik at 
\[
\V v^* A \V v=
\begin{bmatrix}
\frac{1}{\sqrt{2}}   & \frac{1}{\sqrt{2}}    
\end{bmatrix}
\begin{bmatrix}
2     & 1 \\
1      & 2
\end{bmatrix}
\begin{bmatrix}
\frac{1}{\sqrt{2}}    \\
\frac{1}{\sqrt{2}}    
\end{bmatrix}
=3. \qedhere
\]
\end{ex}

\section*{Symmetriske matriser}
\begin{defnx}
En kompleks matrise sies å være 
\defterm{symmetrisk} dersom $A=A^*$. 
\end{defnx}

\begin{merkx}
Dersom $A$ er reell er $A^*=A^T$, slik at kravet for en symmetrisk matrise er at $A=A^T$. 
I litteraturen er det vanlig å reservere begrepet \textit{symmetrisk} for reelle matriser der $A=A^T$, mens komplekse matriser kalles $\textit{hermittiske}$ hvis $A=A^*$. 
Nå er det imidlertid ingen gode grunner til dette, for det er ingen som noen gang transponerer en kompleks matrise. 
\end{merkx}

\begin{ex}
Matrisen
\[
\begin{bmatrix}
1 & 1+i & -i\\  1-i &0 & 2-i \\ i & 2+i  & 2
\end{bmatrix}
\]
er symmetrisk.
\end{ex}

\begin{merkx}
En symmetrisk matrise må har reelle diagonalelementer, siden $z=\overline z$ for alle elementer der.
\end{merkx}

\begin{defnx}
En $n \times n$-matrise er \defterm{ortogonalt diagonaliserbar} dersom den har $n$ ortogonale egenvektorer.
\end{defnx}

\begin{merkx}
Denne definisjonen kommer kanskje som troll i eske. Men alt blir klart om litt.
\end{merkx}

Dersom vi lar $V$ være en $n \times n$-matrise bestånde av $A$ sine $n$ ortonormale egenvektorer, 
ser vi at 
\[
V^* A V =V^* VD =D
\]
siden $V^* V=I$. Motsatt ser vi at dersom
\[
V^* A V =D,
\]
for en ortonormal matrise $V$, utgjør $V$ sine kolonner  $n$ ortonormale egenvektorer for $D$.

\begin{thm}
Vi kan skrive 
\[
V^*AV=D
\]
hvis og bare hvis $A$ har $n$ ortogonale egenvektorer. 
\end{thm}


\begin{thm}
Dersom  $A=A^*$, er 
$
\V x^* A \V x
$
reell.
\end{thm}
\begin{proof}
Vi tar beviset kun for $2 \times 2$-matriser. Selv Gilbert Strang gjør det for sine studenter, og han er professor på Harvard.
En symmetrisk $2 \times 2$-matrise kan skrives
\[
A=\begin{bmatrix}
r_1 & z\\
\overline{z} & r_2
\end{bmatrix}
\]
der $r_1$ og $r_2$ er relle tall. Vi beregner
\[
\V x^* A \V x=r_1x_1\overline{x_1}+z\overline{ x_1}x_2+\overline{z}x_1\overline{ x_2}+r_2x_2\overline{x_2}.
\]
Nå er $r_1x_1\overline{x_1}$ og $r_2x_2\overline{x_2}$ reelle tall, 
mens $z \overline{x_1}x_2$ og 
$\overline{z}x_1\overline{x_2}$ er hverandres komplekskonjugater, 
og følgelig er 
$z\overline{x_1}x_2+\overline{z}x_1\overline{x_2}$ også et reelt tall.
\end{proof}


\begin{thm}
En  symmetrisk matrise har reelle egenverdier.
\end{thm}
\begin{proof}
La $\V v$ være en normalisert egenvektor med egenverdi $\lambda$. Vi vet at 
\[
\V v^* A \V v=\lambda,
\]
og venstresiden er reell, så da må også $\lambda$ være det.
\end{proof}



\begin{thm}
Egenvektorene til to distinkte egenverdier er ortogonale for symmetriske matriser.
\end{thm}

\begin{proof}
La $\V v_1$ og $\V v_2$ være to egenvektorer med egenverdier $\lambda_1$ og $\lambda_2$. 
Vi beregner (husk at $\lambda_1$ og $\lambda_2$ er reelle)
\begin{align*}
\lambda_1 \V v_1^*\V v_2&=(\lambda_1 \V v_1)^*\V v_2=(A \V v_1)^*\V v_2\\&= \V v_1^*A^*\V v_2=\V v_1^*(A\V v_2)=\V v_1^*(\lambda_2\V v_2)=\lambda_2 \V v_1^*\V v_2
\end{align*}
Vi vet altså at
\[
0= \lambda_1 \V v_1^*\V v_2-\lambda_2 \V v_1^*\V v_2=(\lambda_1-\lambda_2) \V v_1^*\V v_2,
\]
og hvis vi bruker at $\lambda_1$ og $\lambda_2$ er forskjellige, må vi ha $\V v_1^*\V v_2=0$.
\end{proof}

\noindent Det siste teoremet vi tar med, må vi dessverre la stå ubevist. Det er litt for hardt.
\begin{thm}
En  symmetrisk matrise er ortogonalt diagonaliserbar.
\end{thm}

\begin{ex}
Matrisen 
\[
\begin{bmatrix}
1 & 2 & 2\\  2 &6 & 2 \\ 2 & 2 & 6
\end{bmatrix}
\]
har egenverdier 4, 9 og 0. Egenvektorerer er henholdsvis
\[
\begin{bmatrix}
0 \\ -1 \\ 1
\end{bmatrix},
\;
\begin{bmatrix}
1 \\ 2 \\ 2
\end{bmatrix}
\quad \text{og} \quad
\begin{bmatrix}
-4 \\ 1 \\ 1
\end{bmatrix}.
\]
Merk at alle vektorer er innbyrdes ortogonale, og matrisen er følgelig ortogonalt diagonaliserbar, med
\[
V=
\begin{bmatrix}
0 & \frac{1}{3} & -\frac{4}{3\sqrt{2}}\\  -\frac{1}{\sqrt{2}} &\frac{2}{3} &  \frac{1}{3\sqrt{2}} \\ \frac{1}{\sqrt{2}} & \frac{2}{3} &  \frac{1}{3\sqrt{2}}.
\end{bmatrix}
\]
og 
\[
D=
\begin{bmatrix}
4 & 0 & 0\\  0 &9 & 0 \\ 0 & 0 & 0
\end{bmatrix}. \qedhere
\]
\end{ex}

\begin{merkx}
Det forrige eksemplet viser at en matrise kan være diagonaliserbar uten å være inverterbar.
\end{merkx}


%\begin{proof}
%Vi ser lett at en symmertrisk $n\times n$-matrise like mange ortogonale egenvektorer som distinkte egenverdier, og at vi kan finne en ortogonal basis for repeterte egenverdier.
%\end{proof}





\kapittelslutt
